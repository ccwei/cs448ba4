\documentclass{sigchi}

% Remove or comment out these two lines for final version
% \toappearbox{\Large Submitted to CHI'13. \\Do not cite, do not circulate.}
\pagenumbering{arabic}% Arabic page numbers for submission. 

% Use \toappear{...} to override the default ACM copyright statement (e.g. for preprints).

% Load basic packages
\usepackage{balance}  % to better equalize the last page
\usepackage{graphics} % for EPS, load graphicx instead
\usepackage{times}    % comment if you want LaTeX's default font
\usepackage{url}      % llt: nicely formatted URLs

% llt: Define a global style for URLs, rather that the default one
\makeatletter
\def\url@leostyle{%
  \@ifundefined{selectfont}{\def\UrlFont{\sf}}{\def\UrlFont{\small\bf\ttfamily}}}
\makeatother
\urlstyle{leo}


% To make various LaTeX processors do the right thing with page size.
\def\pprw{8.5in}
\def\pprh{11in}
\special{papersize=\pprw,\pprh}
\setlength{\paperwidth}{\pprw}
\setlength{\paperheight}{\pprh}
\setlength{\pdfpagewidth}{\pprw}
\setlength{\pdfpageheight}{\pprh}

% Make sure hyperref comes last of your loaded packages, 
% to give it a fighting chance of not being over-written, 
% since its job is to redefine many LaTeX commands.
\usepackage[pdftex]{hyperref}
\hypersetup{
pdftitle={SIGCHI Conference Proceedings Format},
pdfauthor={LaTeX},
pdfkeywords={SIGCHI, proceedings, archival format},
bookmarksnumbered,
pdfstartview={FitH},
colorlinks,
citecolor=black,
filecolor=black,
linkcolor=black,
urlcolor=black,
breaklinks=true,
}

% create a shortcut to typeset table headings
\newcommand\tabhead[1]{\small\textbf{#1}}


% End of preamble. Here it comes the document.
\begin{document}

\title{Peereviz: An Interactive Peer-Review Visualization}

% Note that submissions are blind, so author information should be omitted
\numberofauthors{3}
\author{
  \alignauthor Chih-Chiang Wei\\
    \affaddr{Affiliation}\\
    \affaddr{Address}\\
    \email{e-mail address}\\
    \affaddr{Optional phone number}
  \alignauthor Kanit Wongsuphasawat\\
    \affaddr{Affiliation}\\
    \affaddr{Address}\\
    \email{e-mail address}\\
    \affaddr{Optional phone number}    
  \alignauthor Thiraphat Charoensripongsa\\
    \affaddr{Computer Science Department}\\
    \affaddr{Stanford University}\\
    \email{tchar@stanford.edu}    
}

% Teaser figure can go here
%\teaser{
%  \centering
%  \includegraphics{Figure1}
%  \caption{Teaser Image}
%  \label{fig:teaser}
%}

\maketitle

\begin{abstract}

% rephase this... let's abstract away vecture lab first!
Venture Lab is a Massive Online Open Classroom (MOOC) platform on which students do team projects and peer review for each other's assignments. With this large scale, the data from peer review process is huge and it becomes very difficult to explore the data and understand. Our goal is to design a visualization tool to explore peer review data that helps course instructors and platform designers understand and gain insights from the peer review activities, the engagement of students, and the quality of the reviews. We utilize existing text visualization techniques including tag clouds as well as multiview coordination to visualize the peer evaluation score distribution and the text feedbacks.

\end{abstract}

\keywords{
	Peer Review; MOOC; Text Visualization; MORE;
}	
	% \\\textcolor{red}{Mandatory section to be included in your final version.}
% }

% \category{H.5.m.}{Information Interfaces and Presentation (e.g. HCI)}{Miscellaneous
% \\
% \textcolor{red}{See: \url{http://www.acm.org/about/class/1998/}
% for more information and the full list of ACM classifiers and descriptors. 
% Mandatory section: On the submission page
% only the classifiers' letter-number combination will need to be entered.}
% }

% \terms{
% 	Human Factors; Design; Measurement. 
% 	If you choose more than one ACM General Term, 
% 	separate the terms with a semi-colon.
% \\
% \textcolor{red}{If you choose more than one ACM General Term, 
% separate the terms with a semi-colon. See list of ACM terms at:
% \url{http://www.sheridanprinting.com/sigchi/generalterms.htm}.
% Optional section to be included in your final version.}
% }

\section{Introduction}
% \url{http://www.sheridanprinting.com/info.html}

\section{Related Work}

\section{Venture Desc}

\section{Visualization Design}

\subsection{Layout}

\subsection{Interaction}

% this can be usage observation (http://www.danah.org/papers/InfoViz2005.pdf)

\section{Evaluation}


\section{Future Work}

\section{Conclusion}

\section{Acknowledgments}
Farnaz, thanks a lot!

% Balancing columns in a ref list is a bit of a pain because you
% either use a hack like flushend or balance, or manually insert
% a column break.  http://www.tex.ac.uk/cgi-bin/texfaq2html?label=balance
% multicols doesn't work because we're already in two-column mode,
% and flushend isn't awesome, so I choose balance.  See this
% for more info: http://cs.brown.edu/system/software/latex/doc/balance.pdf
%
% Note that in a perfect world balance wants to be in the first
% column of the last page.
%
% If balance doesn't work for you, you can remove that and
% hard-code a column break into the bbl file right before you
% submit:
%
% http://stackoverflow.com/questions/2149854/how-to-manually-equalize-columns-
% in-an-ieee-paper-if-using-bibtex
%
% Or, just remove \balance and give up on balancing the last page.
%
\balance

% If you want to use smaller typesetting for the reference list,
% uncomment the following line:
% \small
\bibliographystyle{acm-sigchi}
\bibliography{peerevis}
\end{document}
