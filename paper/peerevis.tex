\documentclass{sigchi}

% Remove or comment out these two lines for final version
% \toappearbox{\Large Submitted to CHI'13. \\Do not cite, do not circulate.}
\pagenumbering{arabic}% Arabic page numbers for submission. 

% Use \toappear{...} to override the default ACM copyright statement (e.g. for preprints).

% Load basic packages
\usepackage{balance}  % to better equalize the last page
\usepackage{graphics} % for EPS, load graphicx instead
\usepackage{times}    % comment if you want LaTeX's default font
\usepackage{url}      % llt: nicely formatted URLs

% llt: Define a global style for URLs, rather that the default one
\makeatletter
\def\url@leostyle{%
  \@ifundefined{selectfont}{\def\UrlFont{\sf}}{\def\UrlFont{\small\bf\ttfamily}}}
\makeatother
\urlstyle{leo}


% To make various LaTeX processors do the right thing with page size.
\def\pprw{8.5in}
\def\pprh{11in}
\special{papersize=\pprw,\pprh}
\setlength{\paperwidth}{\pprw}
\setlength{\paperheight}{\pprh}
\setlength{\pdfpagewidth}{\pprw}
\setlength{\pdfpageheight}{\pprh}

% Make sure hyperref comes last of your loaded packages, 
% to give it a fighting chance of not being over-written, 
% since its job is to redefine many LaTeX commands.
\usepackage[pdftex]{hyperref}
\hypersetup{
pdftitle={SIGCHI Conference Proceedings Format},
pdfauthor={LaTeX},
pdfkeywords={SIGCHI, proceedings, archival format},
bookmarksnumbered,
pdfstartview={FitH},
colorlinks,
citecolor=black,
filecolor=black,
linkcolor=black,
urlcolor=black,
breaklinks=true,
}

% create a shortcut to typeset table headings
\newcommand\tabhead[1]{\small\textbf{#1}}


% End of preamble. Here it comes the document.
\begin{document}

\title{Peereviz: Visualizing Peer Reviews}

% Note that submissions are blind, so author information should be omitted
\numberofauthors{3}
\author{
  \alignauthor Chih-Chiang Wei\\
    \affaddr{Affiliation}\\
    \affaddr{Address}\\
    \email{e-mail address}\\    
  \alignauthor Kanit Wongsuphasawat\\
    \affaddr{Affiliation}\\
    \affaddr{Address}\\
    \email{e-mail address}\\    
  \alignauthor Thiraphat Charoensripongsa\\
    \affaddr{Computer Science Department}\\
    \affaddr{Stanford University}\\
    \email{tchar@stanford.edu}    
}

% Teaser figure can go here
%\teaser{
%  \centering
%  \includegraphics{Figure1}
%  \caption{Teaser Image}
%  \label{fig:teaser}
%}

\maketitle

\begin{abstract}

% rephase this... let's abstract away vecture lab first!
Peer reviews are important in project-based classes. In the massive online open classroom (MOOC), the data from peer review process is huge and it becomes very difficult to explore the data and understand. Our goal is to design a visualization tool for peer review exploration which helps course instructors and platform designers understand and gain insights from the peer review activities, the engagement of students, and the quality of the reviews. We utilize existing text visualization techniques as well as multiview coordination to visualize and navigate through score distributions and the text feedbacks.

\end{abstract}

\keywords{
	peer review; group collaboration; visualization; MOOC; exploration
}	
	% \\\textcolor{red}{Mandatory section to be included in your final version.}
% }

% \category{H.5.m.}{Information Interfaces and Presentation (e.g. HCI)}{Miscellaneous
% \\
% \textcolor{red}{See: \url{http://www.acm.org/about/class/1998/}
% for more information and the full list of ACM classifiers and descriptors. 
% Mandatory section: On the submission page
% only the classifiers' letter-number combination will need to be entered.}
% }

% \terms{
% 	Human Factors; Design; Measurement. 
% 	If you choose more than one ACM General Term, 
% 	separate the terms with a semi-colon.
% \\
% \textcolor{red}{If you choose more than one ACM General Term, 
% separate the terms with a semi-colon. See list of ACM terms at:
% \url{http://www.sheridanprinting.com/sigchi/generalterms.htm}.
% Optional section to be included in your final version.}
% }

\section{Introduction}
A massive open online course has just begun its breakthrough in recent years.

....

why peer review important?

..
Our goal is to design a visualization tool for exploring peer review data so that the course instructor 

The paper organization is as follows. We first summarize previous work in peer review visualization. We then describe the Venture Lab platform and the peer review process and feedback data. In the next section, we present our design and visualization techniques, followed by the evaluation and feedbacks from [course instructor/designer?]. Finally, we suggest the possible directions for future work and then summarize our work.

\section{Related Work}
describe: An Interactive Analytic Tool for Peer-Review Exploration
How are we different?: focus on multiview, navigation, qualitative + quantitative feedback

\section{Venture Lab Description}
What's venture lab? - project based, large scale collaboration

The group selection algorithm

The review mechanics: 
- quantitative score 1-10
- qualitative feedback capture grid,

What course instructor do normally to understand the peer review data?
  problem: 
    -slow have to read one by one 
    -hard to see the patterns


\section{Visualization Design}

\subsection{Layout}
-team filters 
-score distribution (histogram)
-teams list
-keywords (see patterns, sense of quality)
  - Unigram, Bigram List
    - color encoding for nominal types (notable, ideas, constructive, question)
  - Tag Cloud
-Reviews
  - Score
  - Feedback capture
  - Text Search


\subsection{Interaction}

-team selection
  - from histogram
    - select 1 team
    - brushing multiple teams
  - Team list
    - show team brief descripton
    - select 1 team
  - Aggregate view
  - search 
    - highlight the histogram
    - update the list
    - update the right panel 

- Search
  -team search
  - keyword search
    -in keyword/phrase list
    -in reviews: aggregate view

%\subsection{Search}


% this can be 'Usage Observation' (http://www.danah.org/papers/InfoViz2005.pdf)
\section{Evaluation}
- Disscuss with Farnaz?
- Platform creator: Joseph?

\section{Future Work}
- generalize for other platforms
- improve language feature: stemming, similar words

\section{Conclusion}
...

\section{Acknowledgments}
Farnaz, thanks a lot!

% Balancing columns in a ref list is a bit of a pain because you
% either use a hack like flushend or balance, or manually insert
% a column break.  http://www.tex.ac.uk/cgi-bin/texfaq2html?label=balance
% multicols doesn't work because we're already in two-column mode,
% and flushend isn't awesome, so I choose balance.  See this
% for more info: http://cs.brown.edu/system/software/latex/doc/balance.pdf
%
% Note that in a perfect world balance wants to be in the first
% column of the last page.
%
% If balance doesn't work for you, you can remove that and
% hard-code a column break into the bbl file right before you
% submit:
%
% http://stackoverflow.com/questions/2149854/how-to-manually-equalize-columns-
% in-an-ieee-paper-if-using-bibtex
%
% Or, just remove \balance and give up on balancing the last page.
%
\balance

% If you want to use smaller typesetting for the reference list,
% uncomment the following line:
% \small
\bibliographystyle{acm-sigchi}
\bibliography{peerevis}
\end{document}
